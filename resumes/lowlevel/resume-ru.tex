\def\LANG{ru}

\documentclass[../../main.tex]{subfiles}

\begin{document}

    \begin{center}
        \bb{\myName}\\[0.5\baselineskip]
        \myBirthDateAndAge\\[\baselineskip]
    \end{center}

    \noindent\begin{tabularx}{\textwidth}{>{\bfseries} p{3cm} X}
        Контактная информация &
            \begin{description}
                \item[Домашний адрес:] \myAddress
                \item[e-mail:] \email{\myEmail}
                \item[Телефон:] \myPhone
            \end{description}\\
        \meta{Образование}
            \submeta{\myBachelorEducationPeriod} & \myBachelorEducation\\
            \submeta{\myMasterEducationPeriod} & \myMasterEducation\\
        Опыт работы & Отсутствует\\
        Языки &
            Русский (родной), технический английский (\en{B1})\\
        Профессио\-наль\-ные навыки &
            \begin{description}
                \item[Языки программирования:] \en{Java} (\myJavaExperience), \en{C/C++} (\myCppExperience), \en{C\#} (\myCsharpExperience), \en{Wolfram Mathematica} (\myWolframExperience), {\LaTeX} (\myTexExperience)
                \item[\ii{Замечание}:] периоды в скобках перекрываются, указан срок \ii{активного} использования технологий
                \item[Операционные системы:] \en{Microsoft Windows}, семейство \en{Unix/""Linux}
                \item[Инструменты:] \en{git}
                \item[Общие навыки:] структуры данных, алгоритмы, шаблоны проектирования, многопоточное программирование, ООП
            \end{description}\\
        О себе &
            \begin{description}
                \item[Области профессионального интереса:] системное программирование, многопоточное и распределенное программирование, модели памяти, формальные системы и основания математики, формальная верификация, функциональное программирование, компиляция и оптимизация, обратная разработка
            \end{description}\\
        Прочее &
            \begin{description}
                \item[GitHub:] \url{\myGithub}
                \item[GitLab:] \url{\myGitlab}
            \end{description}\\
    \end{tabularx}

\end{document}
