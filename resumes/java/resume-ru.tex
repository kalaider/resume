\def\LANG{ru}

\documentclass[../../main.tex]{subfiles}

\begin{document}

    \blockHeader

    \noindent\begin{tabularx}{\textwidth}{>{\bfseries} p{3cm} X}
        Контактная информация & \blockContactInfo\\
        \metablockEducation\\
        Опыт работы & Отсутствует\\
        Языки & \blockLanguages\\
        Профессио\-наль\-ные навыки &
            \begin{description}
                \item[Языки программирования:] \en{Java} (\myJavaExperience), \en{C/C++} (\myCppExperience), \en{C\#} (\myCsharpExperience)
                \item[\ii{Замечание}:] периоды в скобках перекрываются, указан срок \ii{активного} использования технологий
                \item[Прочие языки:] \en{SQL}, \en{HTML}, \en{CSS}
                \item[Протоколы:] \en{TCP/""IP}, \en{HTTP}
                \item[Операционные системы:] \en{Microsoft Windows}, семейство \en{Unix/""Linux}
                \item[Инструменты:] \en{git}, \en{Maven}
                \item[\en{Java}:] \en{Java SE}, \en{Java EE} (включая \en{JPA}, \en{JMS}, \en{JSP}, \en{Bean Validation}), \en{JNI}
                \item[Фреймворки и технологии (\en{Java}):] \en{Spring}, \en{Spring Boot}, \en{Hibernate}
                \item[Серверы/контейнеры приложений:] \en{Tomcat}
                \item[Инструменты тестирования (\en{Java}):] \en{JUnit}, \en{Mockito}
                \item[\en{DevOps}:] основы \en{Docker}, \en{Jenkins}, \en{CI/CD}
                \item[Общие навыки:] структуры данных, алгоритмы, шаблоны проектирования (в т.ч. \en{enterprise}), многопоточное программирование, ООП, \en{UML}
            \end{description}\\
        О себе &
            \begin{description}
                \item[Области профессионального интереса:] системное программирование, многопоточное и распределенное программирование, функциональное программирование, компиляция и оптимизация, обратная разработка
                \item[Обучаемость:] склонность к непрерывному самообучению, возможность легкого освоения новых технологий и языков программирования, в т.ч. функциональных
            \end{description}\\
        Прочее &
            \begin{description}
                \item[GitHub:] \url{\myGithub}
                \item[GitLab:] \url{\myGitlab}
            \end{description}\\
    \end{tabularx}

\end{document}
